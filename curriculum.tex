% Setting up the document class and necessary packages
\documentclass[a4paper,12pt]{article}
\usepackage[utf8]{inputenc}
\usepackage[thai]{babel}
\usepackage{geometry}
\geometry{margin=1in}
\usepackage{enumitem}
\usepackage{booktabs}
\usepackage{titlesec}
\usepackage{amsmath}
\usepackage{xcolor}
% Configuring fonts
\usepackage{noto}

% Customizing section titles
\titleformat{\section}{\Large\bfseries}{\thesection}{1em}{}
\titleformat{\subsection}{\large\bfseries}{\thesubsection}{1em}{}

% Beginning the document
\begin{document}

% Creating the title page
\begin{titlepage}
    \centering
    \vspace*{2cm}
    {\Huge\bfseries หลักสูตรวิชาคอมพิวเตอร์ธุรกิจ พ.ศ. 2568\\}
    \vspace{1cm}
    {\Large บริหารธุรกิจบัณฑิต (คอมพิวเตอร์ธุรกิจดิจิทัล)\\}
    \vspace{0.5cm}
    {\large มหาวิทยาลัยตัวอย่าง\\}
    \vspace{2cm}
    {\large ปีการศึกษา 2568 (ค.ศ. 2025)\\}
    \vfill
\end{titlepage}

% Adding table of contents
\tableofcontents
\newpage

% Introducing the curriculum overview
\section{ภาพรวมหลักสูตร}
หลักสูตรบริหารธุรกิจบัณฑิต สาขาวิชาคอมพิวเตอร์ธุรกิจดิจิทัล พ.ศ. 2568 ออกแบบเพื่อผลิตบัณฑิตที่มีความรู้และทักษะในการประยุกต์ใช้เทคโนโลยีดิจิทัลกับการบริหารธุรกิจ โดยเน้นการพัฒนาระบบสารสนเทศ การวิเคราะห์ข้อมูล และการจัดการธุรกิจในยุคดิจิทัลที่ขับเคลื่อนด้วยปัญญาประดิษฐ์และบล็อกเชน

% Listing objectives
\subsection{วัตถุประสงค์}
\begin{enumerate}
    \item พัฒนาทักษะการเขียนโปรแกรมและการออกแบบระบบสำหรับงานธุรกิจ
    \item สร้างความเข้าใจในเทคโนโลยีสมัยใหม่ เช่น AI, IoT และ Blockchain
    \item เสริมสร้างความสามารถในการวิเคราะห์ข้อมูลเพื่อการตัดสินใจทางธุรกิจ
    \item ปลูกฝังจรรยาบรรณและความรับผิดชอบในงานเทคโนโลยี
\end{enumerate}

% Detailing course structure
\section{โครงสร้างหลักสูตร}
หลักสูตรประกอบด้วย 130 หน่วยกิต แบ่งเป็น:
\begin{itemize}
    \item หมวดวิชาศึกษาทั่วไป: 30 หน่วยกิต
    \item หมวดวิชาเฉพาะบังคับ: 72 หน่วยกิต
    \item หมวดวิชาเลือก: 15 หน่วยกิต
\end{itemize}

% Listing core courses
\subsection{รายวิชาหลัก}
\begin{table}[h]
    \centering
    \begin{tabular}{p{4cm}p{8cm}c}
        \toprule
        \textbf{รหัสวิชา} & \textbf{ชื่อวิชา} & \textbf{หน่วยกิต} \\
        \midrule
        BC2578-101 & การเขียนโปรแกรมเชิงวัตถุขั้นสูง & 3 \\
        BC2578-102 & ระบบฐานข้อมูลอัจฉริยะ & 3 \\
        BC2578-103 & การจัดการพาณิชย์อิเล็กทรอนิกส์ยุค Web 5.0 & 3 \\
        BC2578-104 & ปัญญาประดิษฐ์เพื่อการวิเคราะห์ธุรกิจ & 3 \\
        BC2578-105 & การออกแบบและพัฒนาแอปพลิเคชันควอนตัม & 3 \\
        BC2578-106 & ความมั่นคงปลอดภัยไซเบอร์สำหรับธุรกิจ & 3 \\
        BC2578-107 & โครงงานคอมพิวเตอร์ธุรกิจดิจิทัล & 6 \\
        \bottomrule
    \end{tabular}
    \caption{ตัวอย่างรายวิชาหลัก}
\end{table}

% Describing elective courses
\subsection{รายวิชาเลือก}
\begin{itemize}
    \item การพัฒนาแอปพลิเคชัน AR/VR สำหรับการตลาด
    \item เทคโนโลยีบล็อกเชนในซัพพลายเชน
    \item การจัดการข้อมูลขนาดใหญ่ (Big Data Management)
\end{itemize}

% Outlining expected outcomes
\section{ผลลัพธ์ที่คาดหวัง}
บัณฑิตจะสามารถ:
\begin{itemize}
    \item ออกแบบและพัฒนาระบบสารสนเทศที่ตอบสนองความต้องการขององค์กร
    \item ใช้เครื่องมือ AI และการวิเคราะห์ข้อมูลในการตัดสินใจเชิงกลยุทธ์
    \item จัดการโครงการเทคโนโลยีด้วยความเป็นมืออาชีพ
\end{itemize}

% Ending the document
\end{document}